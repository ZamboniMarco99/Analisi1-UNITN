\chapter{Infinito}
Una funzione $f:X\rightarrow \mathbb{R}, x_0\in\bar{\mathbb{R}}$ punto di accumulazione. Se $\lim\limits_{x\rightarrow x_0}f(x)=\pm\infty$ si dice divergente positivamente o 
negativamente.
\section{Confronto di infiniti}
Dette $f,g,x_0\in\bar{\mathbb{R}}$ punto di accumulazione. Se f e g sono entrambe infinite e $\lim\limits_{x\rightarrow x_0}\frac{f(x)}{g(x)}=0$ f \`e un infinito di ordine 
inferiore rispetto a g. Se invece $\lim\limits_{x\rightarrow x_0}\frac{f(x)}{g(x)}=l, l\in\bar{\mathbb{R}}\backslash 0$, f e g sono infiniti dello stesso ordine, in particolare, se
$l=1$ allora si usa la notazione $f$\textasciitilde$g$ e si dice che f \`e asintotico a g per x che tende a $x_0$. Se invece $\lim\limits_{x\rightarrow x_0}\frac{f(x)}{g(x)}=\infty$
allora si dice che f \`e un infinito di un ordine superiore rispetto a g. Se il limite non esiste allora si dicono non confrontabili.   
\section{Infinitesimi}
$f(x)\rightarrow l $ per $x\rightarrow x_0\Leftrightarrow f(x)-l=o(1)$. Quando indico con o(1) l'infinitesimo, ovvero un valore pi\`u piccolo vicino a 0.
\subsection{Regole di calcolo per o(1)}
\begin{itemize}
\item $ko(1)=o(1),\forall k\in\mathbb{R}$
\item $o(1)o(1)=o(1)$
\item $o(1)+o(1)=o(1)$
\item $(1)-o(1)=o(1)$
\end{itemize}
\section{Confronto di infinitesimi}
Dette $f,g,x_0\in\bar{\mathbb{R}}$ punto di accumulazione. Se f e g sono entrambe infinitesime e $g(x)\neq $ e $\lim\limits_{x\rightarrow x_0}\frac{f(x)}{g(x)}=0$ si dice che f \`e 
un infinitesimo di ordine superiore g, se vale $\infty$ \`e di ordine inferiore.  $\lim\limits_{x\rightarrow x_0}\frac{f(x)}{g(x)}=l$ sono dello stesso ordine e se $l=1$ f e g sono 
asintotiche. Se non esistono i limiti i due non sono confrontabili.